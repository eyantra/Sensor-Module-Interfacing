\documentclass[a4paper,29.6pt]{article}
\usepackage{graphicx}

\title {Script\\Sensor Module Interfacing \\[10pt] Task: RFID Module Interfacing \\[25pt] Team members }
\author {Chayatan \and Mukilan A \and Shanthanu Senguptha}

\begin{document}
\maketitle
\begin{center}
\begin{large}
Under the guidance of\\
\textbf{Prof. Kavi Arya\\and\\Parin Chedda}\\
\vspace{0.5in}
\end{large}
\end{center}
\begin{center}
\includegraphics[scale=0.32]{iitblogo.pdf}
\end{center}
\begin{center}
\begin{large}
Embedded and Real-Time Systems Laboratory \\
Department of Computer Science and Engineering \\
Indian Institute of Technology \\
Bombay \\
\end{large}
\end{center}




\newpage
\section{Slide1: Introduction}
\begin{small}
Hello everyone, this is a video tutorial on interfacing RFID reader Module on FireBird V robot.\\\\

\end{small}

\section{Slide2: Agenda for Discussion}
\begin{small}

We will be discussing on what exactly is RFID, the need for RFID  and its recent applications.
We will also be discussing on the working principle of RFID.\\
Most importantly let us learn how to interface RFID to FBV by looking into its pin and Circuit diagram.\\
Let us also see how the output of the RFID module can be seen on the Serial Terminal. And its necessary requirements.\\
Finally, lets see the basic interfacing c code for ATmega2560 and a small video at the end demonstrating a small application using RFID midule.

\end{small}

\section{Slide3: What is RFID??}
\begin{small}
First of all let us ask ourselves what is an RFID??
RFID means, Radio Frequency  Identification.\\ It is a module which transfers the data to the microcontroller which is wirelessly read from a passive RFID Tag.\\
This feature helps in identifying various real world objects.\\
One feature that distinguishes RFID with a bar code is, these tags are available as active or passive tags. The active tags have a small memory embedded in them and hence data can be both read and written to it as well.

\end{small}

\section{Slide4: Features of EM-18 reader}
\begin{small}
Let us see some of the features of EM-18 RFID reader module.
\begin{itemize}
\item EM-18 operates on $+5V$ power supply. 
\item Read frequency of the EM-18 reader is 125kHz
\item Maximum read range is 10 cm
\item The serial transmission rate is 9600bps, TTL and RS232 output
\end{itemize}

\end{small}




\section{Slide5: Pin Details}
\begin{small}
Now let us look into the Pin details of EM-18 RFID reader module
\begin{itemize}
\item Pin 1 and 2: These are supply and ground pins
\item Pin 3: The Beep pin is the output pin which provides series of pulses which could be connected to an led or a buzzer for the read indication
\item Pin 4 and 5: These two are antenna pins which are left unconnected.
\item Pin 6: SEL Pin is pulled high to get RS232 output. If the Pin is held low then data is received from DO and D1 pins.
\item Pin 7: The serial output is taken from this pin in RS232 format
\item Pin 8 and 9: These Data signal Pins used to output the data in 26 bit Wiegand format
\end{itemize}
\end{small}


\section{Slide6: USB to Serial Converter}
\begin{small}
Now that we are sure of our connections we can check the output in two ways..\\\\
1.	 Using the Serial Terminal Software with the help of USB to Serial converter\\
2.	Writing the C code and checking the output on the LCD screen.\\\\

Now lets see how to see output on Serial Terminal.\\ Follow these steps to achieve the task\\
\begin{itemize}
\item	The first figure shows the USB to Serial converter
\item	The second figure shows the connections to be made for serial transmission
\item	Connect the common ground - pin of the converter
\item	VCC Pin of the Converter need not be connected
\item	The RS232 output pin is connected to RX pin of the Converter
\end{itemize}

\end{small}

\section{Slide7: Serial Terminal}
\begin{small}
To see the output on the Serial Terminal,
\begin{itemize}
\item	Open the serial terminal software
\item	Set the COM port for the device
\item	Set the baud rate to 9600
\item	Set the number of start bits, stop bits and \item parity bits
\item	Change the Data mode to Text
\end{itemize}

\end{small}


\section{Slide8: Output on LCD Screen}
\begin{small}
This slide shows the Sample output as seen on the LCD screen of the FireBird V robot.\\
Here the module is displaying the Read Tag on the Second line of the LCD Display.\\

\end{small}

\section{Slide9: The C code}
\begin{small}
Now let us see the necessary C Code for the RFID module to function.
\end{small}
\end{document}
